\documentclass[12pt,a4paper]{article}
\usepackage{graphicx}
\usepackage{hyperref}
\usepackage{float}
\usepackage{geometry}
\usepackage{longtable}
\usepackage{booktabs}
\usepackage{array}
\usepackage{calc}
\usepackage{amsmath}
\usepackage{setspace}
\usepackage{times}

% Pandoc compatibility macros
\providecommand{\tightlist}{%
  \setlength{\itemsep}{0pt}\setlength{\parskip}{0pt}}
\newcommand{\real}[1]{#1}

\geometry{
 a4paper,
 total={170mm,257mm},
 left=25mm,
 top=25mm,
 right=20mm,
 bottom=20mm,
}

\date{}

\begin{document}

% -------------------------
% 1. TITLE PAGE
% -------------------------
\begin{titlepage}
    \begin{center}
        \vspace*{1cm}

        \Huge
        \textbf{Secure Chat Project Report}

        \vspace{1.5cm}

        \large
        \textit{A Mini Project Report Submitted in partial fulfillment of the requirements\\ for the degree of}

        \vspace{0.5cm}

        \Large
        \textbf{Bachelor of Technology}

        \vspace{0.5cm}

        \large
        In

        \vspace{0.5cm}

        \Large
        \textbf{Department}

        \vspace{1.5cm}

        \textbf{Submitted by:}

        \vspace{0.5cm}
        \Large
        Student Name \\
        (Roll No: )

        \vspace{1.5cm}

        \large
        \textbf{Under the Guidance of:}

        \vspace{0.5cm}
        \Large
        

        \vfill

        \Large
        \textbf{College Name / University Logo Here} \\
        \large
        City, State - Zip Code \\
        \textbf{2024-2025}

    \end{center}
\end{titlepage}

% -------------------------
% 2. CERTIFICATE
% -------------------------
\newpage
\begin{center}
    \vspace*{2cm}
    \Large \textbf{CERTIFICATE}
    \vspace{1cm}
\end{center}

\noindent This is to certify that the Mini Project entitled \textbf{Secure Chat Project Report} submitted by \textbf{Student Name} (Roll No: ) to the \textbf{Department} of College Name, in partial fulfillment for the award of the degree of \textbf{Bachelor of Technology}, is a bona fide record of work carried out by them under my supervision.

\vspace{3cm}

\noindent
\begin{minipage}{0.45\textwidth}
    \begin{flushleft}
        \textbf{} \\
        (Project Guide)
    \end{flushleft}
\end{minipage}
\hfill
\begin{minipage}{0.45\textwidth}
    \begin{flushright}
        \textbf{Head of Department} \\
        Department
    \end{flushright}
\end{minipage}

\vfill
\begin{center}
    External Examiner \hspace{4cm} Internal Examiner
\end{center}

% -------------------------
% 3. DECLARATION
% -------------------------
\newpage
\begin{center}
    \vspace*{2cm}
    \Large \textbf{DECLARATION}
    \vspace{1cm}
\end{center}

\noindent I hereby declare that the Mini Project entitled \textbf{Secure Chat Project Report} submitted in partial fulfillment of the requirements for the award of the degree of Bachelor of Technology in \textbf{Department} is a record of original work done by me under the supervision of \textbf{}. This project work has not been submitted to any other University or Institute for the award of any degree or diploma.

\vspace{3cm}

\noindent
\begin{flushright}
    \textbf{Student Name} \\
    (Roll No: )
\end{flushright}

% -------------------------
% 4. ACKNOWLEDGMENT
% -------------------------
\newpage
\begin{center}
    \vspace*{2cm}
    \Large \textbf{ACKNOWLEDGMENT}
    \vspace{1cm}
\end{center}

\noindent I would like to express my sincere gratitude to my guide \textbf{} for their continuous support and guidance throughout this Mini Project. I also thank the Head of the Department, Department, and the faculty members for their encouragement.

\vspace{2cm}

\noindent Date: \today

\vspace{2cm}

\noindent
\begin{flushright}
    \textbf{Student Name}
\end{flushright}


% -------------------------
% 5. TABLE OF CONTENTS
% -------------------------
\newpage
\tableofcontents
\newpage

% -------------------------
% 6. MAIN CONTENT
% -------------------------

% Abstract if provided
\section*{Abstract}


\newpage

% Body
\section{\texorpdfstring{\textbf{1.
Introduction}}{1. Introduction}}\label{introduction}

In the present era of rapid digitalization, communication systems have
expanded from physical interactions to internet-based messaging
platforms. Individuals, enterprises, and governments frequently exchange
sensitive and confidential information such as personal identity
details, financial transactions, internal business documents, and
strategic communication. As the volume of digital communication
increases, the risk of unauthorized access, identity theft, data
manipulation, and surveillance also grows significantly. This has led to
the need for \textbf{robust information security mechanisms} that ensure
privacy and trust in communication systems.

Information Security refers to the collection of principles and
techniques used to safeguard data from unauthorized access, disclosure,
disruption, modification, or destruction. The core objective of
information security is to protect the \textbf{Confidentiality,
Integrity, and Availability (CIA)} of data. In messaging applications,
particularly those used for private or organizational communication,
these three pillars become critically important:

\begin{itemize}
\item
  \textbf{Confidentiality:} Ensuring only the intended recipient can
  read the message.
\item
  \textbf{Integrity:} Preventing messages from being modified during
  transmission.
\item
  \textbf{Availability:} Ensuring the communication system operates
  reliably when needed.
\end{itemize}

Traditional communication systems were primarily secured through
physical controls and restricted access networks. However, with the rise
of open networks such as the Internet, \textbf{cyberattacks like
eavesdropping, session hijacking, phishing, spoofing, and
man-in-the-middle attacks} became common. Attackers may intercept
network traffic, alter messages, impersonate users, or monitor user
behavior to obtain sensitive information. Therefore, modern secure
messaging must rely on \textbf{strong cryptographic mechanisms} that
protect data even when transmitted over insecure channels.

\section{\texorpdfstring{\textbf{Need for Secure Messaging
Systems}}{Need for Secure Messaging Systems}}\label{need-for-secure-messaging-systems}

Secure messaging systems are designed to protect digital communication
from unauthorized access. They use encryption to convert the message
into unreadable form, and only the authorized recipient can decode it.
Some of the key reasons secure messaging is necessary are:

\subsubsection{\texorpdfstring{\textbf{1. Protection Against
Eavesdropping}}{1. Protection Against Eavesdropping}}\label{protection-against-eavesdropping}

In unsecured networks, attackers can intercept messages using packet
sniffers or network monitoring tools. End-to-end encryption ensures that
intercepted messages remain unreadable.

\subsubsection{\texorpdfstring{\textbf{2. Prevention of Message
Tampering}}{2. Prevention of Message Tampering}}\label{prevention-of-message-tampering}

Attackers may modify the message during transmission to mislead or
manipulate decisions. Secure messaging systems ensure \textbf{message
integrity}, meaning even a small alteration is detectable.

\subsubsection{\texorpdfstring{\textbf{3. Authentication of
Communicating
Parties}}{3. Authentication of Communicating Parties}}\label{authentication-of-communicating-parties}

It is essential to verify that the person on the other side is who they
claim to be. Secure messaging integrates \textbf{authentication
mechanisms} such as credentials, digital signatures, or public key
certificates.

\subsubsection{\texorpdfstring{\textbf{4. Confidentiality of Sensitive
and Personal
Information}}{4. Confidentiality of Sensitive and Personal Information}}\label{confidentiality-of-sensitive-and-personal-information}

Personal chats, business discussions, and official instructions often
contain confidential data. Unauthorized exposure can result in financial
loss, legal violations, or personal harm.

\subsubsection{\texorpdfstring{\textbf{5. Defense Against Surveillance
and Privacy
Violations}}{5. Defense Against Surveillance and Privacy Violations}}\label{defense-against-surveillance-and-privacy-violations}

In many cases, governments, service providers, or attackers attempt to
track conversations for monitoring. Secure messaging protects users'
fundamental right to privacy.

\subsubsection{\texorpdfstring{\textbf{6. Compliance with Security
Standards}}{6. Compliance with Security Standards}}\label{compliance-with-security-standards}

Industries such as healthcare, banking, and defense must comply with
regulatory standards (e.g., HIPAA, GDPR). Secure messaging helps
organizations meet these legal requirements.

\section{\texorpdfstring{\textbf{Role of Cryptography in Secure
Messaging}}{Role of Cryptography in Secure Messaging}}\label{role-of-cryptography-in-secure-messaging}

The core solution to the above threats lies in \textbf{cryptography}.
Cryptography is the study of techniques used to secure communication by
converting data into protected forms. In secure messaging applications,
cryptography ensures:

\begin{itemize}
\item
  \textbf{Only the intended recipient can read the message}
  (Confidentiality)
\item
  \textbf{Message cannot be altered without detection} (Integrity)
\item
  \textbf{Sender and receiver identities are verified} (Authentication)
\item
  Modern secure messaging systems use a combination of:
\item
  \textbf{Symmetric Key Cryptography (AES)} for fast data encryption
\item
  \textbf{Asymmetric Key Cryptography (RSA, Diffie-Hellman)} for secure
  key exchange
\item
  \textbf{Hashing (SHA-256)} for data integrity
\end{itemize}

This combination is known as \textbf{Hybrid Encryption}, and it is used
in widely adopted apps like \textbf{WhatsApp, Signal, Telegram, and
banking messaging systems}.

\section{\texorpdfstring{\textbf{2. Cryptography
Fundamentals}}{2. Cryptography Fundamentals}}\label{cryptography-fundamentals}

Cryptography is the science of transforming data in such a way that only
authorized individuals can understand and use it. The term originates
from the Greek words \emph{kryptos}, meaning ``hidden,'' and
\emph{graphein}, meaning ``to write.'' In modern computing and
communication environments, cryptography is essential for securing data
transmitted across open and unsecured networks such as the Internet.

Cryptography ensures four major security goals:

{\def\LTcaptype{none} % do not increment counter
\begin{longtable}[]{@{}
  >{\raggedright\arraybackslash}p{(\linewidth - 4\tabcolsep) * \real{0.2683}}
  >{\raggedright\arraybackslash}p{(\linewidth - 4\tabcolsep) * \real{0.3902}}
  >{\raggedright\arraybackslash}p{(\linewidth - 4\tabcolsep) * \real{0.3171}}@{}}
\toprule\noalign{}
\begin{minipage}[b]{\linewidth}\raggedright
\textbf{Property}
\end{minipage} & \begin{minipage}[b]{\linewidth}\raggedright
\textbf{Meaning}
\end{minipage} & \begin{minipage}[b]{\linewidth}\raggedright
\textbf{Importance in Messaging}
\end{minipage} \\
\midrule\noalign{}
\endhead
\bottomrule\noalign{}
\endlastfoot
\textbf{Confidentiality} & Only authorized users can read the data &
Prevents eavesdropping and spying \\
\textbf{Integrity} & Data cannot be altered during transmission &
Prevents tampering attacks \\
\textbf{Authentication} & Verifies the identity of communicating users &
Prevents impersonation \\
\textbf{Non-repudiation} & Sender cannot deny their actions later &
Ensures accountability \\
\end{longtable}
}

To achieve these goals, multiple cryptographic techniques are used,
primarily categorized into \textbf{Symmetric Key Cryptography},
\textbf{Asymmetric Key Cryptography}, and \textbf{Hash Functions}.

\subsection{\texorpdfstring{\textbf{2.1 Symmetric Key
Cryptography}}{2.1 Symmetric Key Cryptography}}\label{symmetric-key-cryptography}

In symmetric key cryptography, the \textbf{same key} is used for
encryption (locking) and decryption (unlocking) of data.

\subsubsection{\texorpdfstring{\textbf{Characteristics}}{Characteristics}}\label{characteristics}

\begin{itemize}
\item
  Single shared key
\item
  Very fast and efficient
\item
  Suitable for encrypting large data (messages, files, streams)
\item
  Requires secure key distribution between sender and receiver
\end{itemize}

\subsubsection{\texorpdfstring{\textbf{Example
Algorithms}}{Example Algorithms}}\label{example-algorithms}

{\def\LTcaptype{none} % do not increment counter
\begin{longtable}[]{@{}
  >{\raggedright\arraybackslash}p{(\linewidth - 4\tabcolsep) * \real{0.3784}}
  >{\raggedright\arraybackslash}p{(\linewidth - 4\tabcolsep) * \real{0.1892}}
  >{\raggedright\arraybackslash}p{(\linewidth - 4\tabcolsep) * \real{0.4054}}@{}}
\toprule\noalign{}
\begin{minipage}[b]{\linewidth}\raggedright
\textbf{Algorithm}
\end{minipage} & \begin{minipage}[b]{\linewidth}\raggedright
\textbf{Key Size}
\end{minipage} & \begin{minipage}[b]{\linewidth}\raggedright
\textbf{Notes}
\end{minipage} \\
\midrule\noalign{}
\endhead
\bottomrule\noalign{}
\endlastfoot
\textbf{AES (Advanced Encryption Standard)} & 128/192/256 bits & Highly
secure, used in modern encryption \\
DES & 56 bits & Outdated due to weakness \\
3DES & 168 bits & More secure than DES but slower \\
\end{longtable}
}

\subsubsection{\texorpdfstring{\textbf{AES (Used in This
Project)}}{AES (Used in This Project)}}\label{aes-used-in-this-project}

AES is a block cipher which processes data in 128-bit blocks.\\
Our system uses \textbf{AES-GCM}, which provides:

\begin{itemize}
\item
  \textbf{Confidentiality} (encrypts data)
\item
  \textbf{Integrity} (via authentication tag)
\end{itemize}

This makes AES-GCM suitable for secure chat applications.

\subsection{\texorpdfstring{\textbf{2.2 Asymmetric Key Cryptography
(Public Key
Cryptography)}}{2.2 Asymmetric Key Cryptography (Public Key Cryptography)}}\label{asymmetric-key-cryptography-public-key-cryptography}

In asymmetric cryptography, \textbf{two different keys} are used:

{\def\LTcaptype{none} % do not increment counter
\begin{longtable}[]{@{}
  >{\raggedright\arraybackslash}p{(\linewidth - 4\tabcolsep) * \real{0.1667}}
  >{\raggedright\arraybackslash}p{(\linewidth - 4\tabcolsep) * \real{0.2361}}
  >{\raggedright\arraybackslash}p{(\linewidth - 4\tabcolsep) * \real{0.2500}}@{}}
\toprule\noalign{}
\begin{minipage}[b]{\linewidth}\raggedright
\textbf{Key}
\end{minipage} & \begin{minipage}[b]{\linewidth}\raggedright
\textbf{Purpose}
\end{minipage} & \begin{minipage}[b]{\linewidth}\raggedright
\textbf{Visibility}
\end{minipage} \\
\midrule\noalign{}
\endhead
\bottomrule\noalign{}
\endlastfoot
\textbf{Public Key} & Used for encryption & Shared openly \\
\textbf{Private Key} & Used for decryption & Kept secret by owner \\
\end{longtable}
}

\subsubsection{\texorpdfstring{\textbf{Benefits}}{Benefits}}\label{benefits}

\begin{itemize}
\item
  No need to share secret keys beforehand
\item
  Enables secure communication with strangers
\item
  Supports \textbf{digital signatures} and \textbf{authentication}
\end{itemize}

\subsubsection{\texorpdfstring{\textbf{RSA (Used for Key Exchange in
This
Project)}}{RSA (Used for Key Exchange in This Project)}}\label{rsa-used-for-key-exchange-in-this-project}

RSA is based on the mathematical difficulty of factoring large prime
numbers.

{\def\LTcaptype{none} % do not increment counter
\begin{longtable}[]{@{}
  >{\raggedright\arraybackslash}p{(\linewidth - 4\tabcolsep) * \real{0.2917}}
  >{\raggedright\arraybackslash}p{(\linewidth - 4\tabcolsep) * \real{0.1389}}
  >{\raggedright\arraybackslash}p{(\linewidth - 4\tabcolsep) * \real{0.3056}}@{}}
\toprule\noalign{}
\begin{minipage}[b]{\linewidth}\raggedright
\textbf{Operation}
\end{minipage} & \begin{minipage}[b]{\linewidth}\raggedright
\textbf{Key Used}
\end{minipage} & \begin{minipage}[b]{\linewidth}\raggedright
\textbf{Purpose}
\end{minipage} \\
\midrule\noalign{}
\endhead
\bottomrule\noalign{}
\endlastfoot
Encrypt AES session key & Public Key & Sent to server securely \\
Decrypt AES session key & Private Key & Server extracts session key \\
\end{longtable}
}

Because RSA is computationally \textbf{slower}, it is used \textbf{only
for exchanging keys}, not encrypting full messages.

\subsection{\texorpdfstring{\textbf{2.3 Hash
Functions}}{2.3 Hash Functions}}\label{hash-functions}

A \textbf{hash function} converts input data into a fixed-size,
irreversible output known as a \textbf{digest} or \textbf{hash value}.

\subsubsection{\texorpdfstring{\textbf{Properties of Good Hash
Function}}{Properties of Good Hash Function}}\label{properties-of-good-hash-function}

\begin{itemize}
\item
  \textbf{One-way}: Cannot derive original message from hash
\item
  \textbf{Collision resistant}: No two messages should produce same hash
\item
  \textbf{Deterministic}: Same input always gives the same hash
\end{itemize}

\subsubsection{\texorpdfstring{\textbf{SHA-256 (Used in this Project for
Password
Hashing)}}{SHA-256 (Used in this Project for Password Hashing)}}\label{sha-256-used-in-this-project-for-password-hashing}

SHA-256 generates a 256-bit hash.

Example:

Input: "hello"Output: 2cf24dba5fb0a... (64 hex characters)

We store \textbf{passwords as hash values}, not raw text increases
security against database leaks.

\subsection{\texorpdfstring{\textbf{2.4 Hybrid Encryption Model (Used in
Our
System)}}{2.4 Hybrid Encryption Model (Used in Our System)}}\label{hybrid-encryption-model-used-in-our-system}

Modern secure messaging applications use a \textbf{hybrid encryption
system} combining both:

{\def\LTcaptype{none} % do not increment counter
\begin{longtable}[]{@{}
  >{\raggedright\arraybackslash}p{(\linewidth - 2\tabcolsep) * \real{0.4444}}
  >{\raggedright\arraybackslash}p{(\linewidth - 2\tabcolsep) * \real{0.3611}}@{}}
\toprule\noalign{}
\begin{minipage}[b]{\linewidth}\raggedright
\textbf{Technique}
\end{minipage} & \begin{minipage}[b]{\linewidth}\raggedright
\textbf{Purpose}
\end{minipage} \\
\midrule\noalign{}
\endhead
\bottomrule\noalign{}
\endlastfoot
\textbf{Asymmetric Encryption (RSA)} & Secure key exchange \\
\textbf{Symmetric Encryption (AES)} & Fast encrypted messaging \\
\end{longtable}
}

\subsubsection{\texorpdfstring{\textbf{Why Hybrid
Encryption?}}{Why Hybrid Encryption?}}\label{why-hybrid-encryption}

{\def\LTcaptype{none} % do not increment counter
\begin{longtable}[]{@{}
  >{\raggedright\arraybackslash}p{(\linewidth - 4\tabcolsep) * \real{0.2083}}
  >{\raggedright\arraybackslash}p{(\linewidth - 4\tabcolsep) * \real{0.2083}}
  >{\raggedright\arraybackslash}p{(\linewidth - 4\tabcolsep) * \real{0.2917}}@{}}
\toprule\noalign{}
\begin{minipage}[b]{\linewidth}\raggedright
\textbf{Factor}
\end{minipage} & \begin{minipage}[b]{\linewidth}\raggedright
\textbf{RSA}
\end{minipage} & \begin{minipage}[b]{\linewidth}\raggedright
\textbf{AES}
\end{minipage} \\
\midrule\noalign{}
\endhead
\bottomrule\noalign{}
\endlastfoot
Speed & Slow & Fast \\
Suitable for & Keys & Messages \\
\end{longtable}
}

So we:

\begin{itemize}
\item
  Generate AES key locally on client
\item
  Encrypt AES key using server\textquotesingle s RSA public key
\item
  Send encrypted AES key to server
\item
  Use AES key for all message encryption
\end{itemize}

This provides:\\
Security\\
Performance\\
End-to-end Encryption Behavior

\subsection{\texorpdfstring{\textbf{2.5 Summary of Cryptographic Use in
the
Project}}{2.5 Summary of Cryptographic Use in the Project}}\label{summary-of-cryptographic-use-in-the-project}

{\def\LTcaptype{none} % do not increment counter
\begin{longtable}[]{@{}
  >{\raggedright\arraybackslash}p{(\linewidth - 4\tabcolsep) * \real{0.2267}}
  >{\raggedright\arraybackslash}p{(\linewidth - 4\tabcolsep) * \real{0.3200}}
  >{\raggedright\arraybackslash}p{(\linewidth - 4\tabcolsep) * \real{0.4267}}@{}}
\toprule\noalign{}
\begin{minipage}[b]{\linewidth}\raggedright
\textbf{Purpose}
\end{minipage} & \begin{minipage}[b]{\linewidth}\raggedright
\textbf{Algorithm Used}
\end{minipage} & \begin{minipage}[b]{\linewidth}\raggedright
\textbf{Reason}
\end{minipage} \\
\midrule\noalign{}
\endhead
\bottomrule\noalign{}
\endlastfoot
Secure key exchange & \textbf{RSA} & Avoids unsafe key sharing \\
Message encryption & \textbf{AES-GCM} & Fast and secure message
confidentiality \\
Password protection & \textbf{bcrypt + SHA-256} & Prevents password
theft \\
Data integrity & Built-in AES-GCM Auth Tag & Detects message
tampering \\
\end{longtable}
}

\section{}\label{section}

\section{}\label{section-1}

\section{}\label{section-2}

\section{}\label{section-3}

\section{}\label{section-4}

\section{}\label{section-5}

\section{}\label{section-6}

\section{}\label{section-7}

\section{}\label{section-8}

\section{\texorpdfstring{\textbf{3. System
Architecture}}{3. System Architecture}}\label{system-architecture}

The Secure Messaging System follows a \textbf{client--server
architecture} in which users (clients) communicate with each other
indirectly through a secure backend server. The server is responsible
for authentication, encrypted key management, controlled message
routing, and intrusion monitoring. The frontend application provides a
user interface to send and receive messages in real-time.

The main components of the system are:

{\def\LTcaptype{none} % do not increment counter
\begin{longtable}[]{@{}
  >{\raggedright\arraybackslash}p{(\linewidth - 4\tabcolsep) * \real{0.2466}}
  >{\raggedright\arraybackslash}p{(\linewidth - 4\tabcolsep) * \real{0.2740}}
  >{\raggedright\arraybackslash}p{(\linewidth - 4\tabcolsep) * \real{0.4521}}@{}}
\toprule\noalign{}
\begin{minipage}[b]{\linewidth}\raggedright
\textbf{Component}
\end{minipage} & \begin{minipage}[b]{\linewidth}\raggedright
\textbf{Technology Used}
\end{minipage} & \begin{minipage}[b]{\linewidth}\raggedright
\textbf{Role in System}
\end{minipage} \\
\midrule\noalign{}
\endhead
\bottomrule\noalign{}
\endlastfoot
\textbf{Frontend UI} & React + Tailwind CSS & Provides interface for
secure chat communication \\
\textbf{Backend Server} & FastAPI (Python) & Handles authentication,
encryption logic, and routing \\
\textbf{Cryptographic Layer} & RSA (2048-bit) + AES-GCM (256-bit) &
Ensures confidentiality and integrity of messages \\
\textbf{Database} & SQLite & Stores user credentials and message
records \\
\textbf{WebSocket Communication} & FastAPI WebSocket API & Enables
real-time encrypted messaging \\
\textbf{IDS Module} & Python-based login monitoring & Detects suspicious
login attempts and blocks attackers \\
\end{longtable}
}

\subsection{\texorpdfstring{\textbf{3.1 Overall Architecture
Diagram}}{3.1 Overall Architecture Diagram}}\label{overall-architecture-diagram}

\begin{figure}
\centering
\includegraphics[width=4.68889in,height=4.05903in,alt={ChatGPT Image Nov 10, 2025, 03\_59\_30 PM}]{/Users/sj/Desktop/devsprint/report-engine/workspace/32a2bd09-5eed-4bc4-8ba4-8d9fc92e7564/media/image1.png}
\caption{ChatGPT Image Nov 10, 2025, 03\_59\_30 PM}
\end{figure}

In this architecture, the server acts not as a message reader, but as a
\textbf{secure relay}, re-encrypting messages for recipients using their
own session keys.

\subsection{\texorpdfstring{\textbf{3.2 Components
Description}}{3.2 Components Description}}\label{components-description}

\subsubsection{\texorpdfstring{\textbf{1. Client (React
UI)}}{1. Client (React UI)}}\label{client-react-ui}

\begin{itemize}
\item
  Runs in the user's browser.
\item
  Generates the AES session key.
\item
  Encrypts plaintext messages before sending.
\item
  Decrypts incoming messages using the shared key.
\item
  Maintains WebSocket connection for real-time message delivery.
\end{itemize}

\subsubsection{\texorpdfstring{\textbf{2. FastAPI
Backend}}{2. FastAPI Backend}}\label{fastapi-backend}

\begin{itemize}
\item
  Serves as the central core logic controller.
\item
  Authenticates users and provides JWT tokens.
\item
  Stores and retrieves encrypted messages.
\item
  Provides the RSA public key to clients.
\item
  Maintains AES session key table per logged-in user.
\end{itemize}

\subsubsection{\texorpdfstring{\textbf{3. Cryptographic
Unit}}{3. Cryptographic Unit}}\label{cryptographic-unit}

\begin{itemize}
\item
  Responsible for securing communication.
\item
  Uses \textbf{RSA (Asymmetric Key)} for exchanging AES keys.
\item
  Uses \textbf{AES-GCM (Symmetric Key)} for encrypting chat messages.
\item
  Ensures message \textbf{confidentiality} and \textbf{integrity}.
\end{itemize}

\subsubsection{\texorpdfstring{\textbf{4. Database
(SQLite)}}{4. Database (SQLite)}}\label{database-sqlite}

Stores:

\begin{itemize}
\item
  Username and password hashes (bcrypt protected)
\item
  Encrypted message records (for offline delivery)
\item
  Login attempt logs (for IDS tracking)
\end{itemize}

\subsubsection{\texorpdfstring{\textbf{5. WebSocket Message
Router}}{5. WebSocket Message Router}}\label{websocket-message-router}

\begin{itemize}
\item
  Maintains persistent real-time connections.
\item
  Delivers encrypted messages instantly to online recipients.
\item
  Eliminates the need for polling.
\end{itemize}

\subsubsection{\texorpdfstring{\textbf{6. Intrusion Detection System
(IDS)}}{6. Intrusion Detection System (IDS)}}\label{intrusion-detection-system-ids}

Monitors failed login attempts per IP address.

Automatically blocks suspicious users.

Helps defend against brute-force and credential stuffing attacks.

\subsection{\texorpdfstring{\textbf{3.3 Detailed Workflow of
System}}{3.3 Detailed Workflow of System}}\label{detailed-workflow-of-system}

\subsubsection{\texorpdfstring{\textbf{Step 1: User
Registration}}{Step 1: User Registration}}\label{step-1-user-registration}

\begin{enumerate}
\def\labelenumi{\arabic{enumi}.}
\item
  User inputs username and password.
\item
  Password is hashed using \textbf{bcrypt}.
\item
  Hashed password is stored in the database.
\end{enumerate}

\subsubsection{\texorpdfstring{\textbf{Step 2: User
Login}}{Step 2: User Login}}\label{step-2-user-login}

\begin{enumerate}
\def\labelenumi{\arabic{enumi}.}
\item
  User submits login credentials.
\item
  Server checks password hash.
\item
  If verified Server returns a \textbf{JWT authentication token}.
\item
  If repeated login failure occurs IDS blocks the IP temporarily.
\end{enumerate}

\subsubsection{\texorpdfstring{\textbf{Step 3: Session Key
Establishment}}{Step 3: Session Key Establishment}}\label{step-3-session-key-establishment}

Client Generates AES Key Encrypts AES Key using Server Public RSA Sends
to Server

Server Decrypts AES Key using Private RSA Stores AES Key for the Session

This ensures that:

\begin{itemize}
\item
  Key exchange is \textbf{confidential}.
\item
  Server does not store user passwords or private cryptographic keys.
\end{itemize}

\subsubsection{\texorpdfstring{\textbf{Step 4: Real-Time
Messaging}}{Step 4: Real-Time Messaging}}\label{step-4-real-time-messaging}

\begin{enumerate}
\def\labelenumi{\arabic{enumi}.}
\item
  Client encrypts message using AES-GCM.
\item
  Message is sent to server via WebSocket or HTTP /send.
\item
  Server decrypts message using sender's AES key.
\item
  Server re-encrypts plaintext using recipient\textquotesingle s AES
  key.
\item
  Server forwards encrypted message instantly to recipient's WebSocket
  client.
\end{enumerate}

\subsubsection{\texorpdfstring{\textbf{Step 5: Message Decryption at
Recipient}}{Step 5: Message Decryption at Recipient}}\label{step-5-message-decryption-at-recipient}

\begin{enumerate}
\def\labelenumi{\arabic{enumi}.}
\item
  Recipient receives encrypted message over WebSocket.
\item
  Recipient decrypts using local AES key.
\item
  Plaintext message is displayed in the UI.
\end{enumerate}

\subsection{\texorpdfstring{\textbf{3.4 Advantages of the
Architecture}}{3.4 Advantages of the Architecture}}\label{advantages-of-the-architecture}

{\def\LTcaptype{none} % do not increment counter
\begin{longtable}[]{@{}
  >{\raggedright\arraybackslash}p{(\linewidth - 2\tabcolsep) * \real{0.3750}}
  >{\raggedright\arraybackslash}p{(\linewidth - 2\tabcolsep) * \real{0.5278}}@{}}
\toprule\noalign{}
\begin{minipage}[b]{\linewidth}\raggedright
\textbf{Feature}
\end{minipage} & \begin{minipage}[b]{\linewidth}\raggedright
\textbf{Benefit}
\end{minipage} \\
\midrule\noalign{}
\endhead
\bottomrule\noalign{}
\endlastfoot
Hybrid Encryption & Combines performance and strong security \\
Session-based AES keys & Allows unique encryption per user \\
Server does not store plaintext & Protects privacy even if server is
compromised \\
Real-time WebSocket messaging & Faster and more efficient than
polling \\
IDS Integration & Enhances system resilience against attackers \\
\end{longtable}
}

\section{\texorpdfstring{\textbf{4. Key Exchange Mechanism (RSA + AES
Hybrid Encryption
Model)}}{4. Key Exchange Mechanism (RSA + AES Hybrid Encryption Model)}}\label{key-exchange-mechanism-rsa-aes-hybrid-encryption-model}

Secure communication requires that both parties share a secret
encryption key that cannot be intercepted or derived by attackers.
Directly transmitting this key over the network is dangerous because
attackers may intercept it. To solve this problem, modern secure
messaging systems use a \textbf{hybrid encryption model} that combines
\textbf{Asymmetric Encryption (RSA)} and \textbf{Symmetric Encryption
(AES)}.

\begin{itemize}
\item
  \textbf{RSA (Public-Key Cryptography)} is used to securely exchange
  the AES key.
\item
  \textbf{AES (Symmetric-Key Cryptography)} is used to encrypt and
  decrypt actual chat messages.
\end{itemize}

This ensures both \textbf{security} and \textbf{high performance}.

\subsection{\texorpdfstring{\textbf{4.1 Why Hybrid Encryption is
Needed}}{4.1 Why Hybrid Encryption is Needed}}\label{why-hybrid-encryption-is-needed}

{\def\LTcaptype{none} % do not increment counter
\begin{longtable}[]{@{}
  >{\raggedright\arraybackslash}p{(\linewidth - 4\tabcolsep) * \real{0.2083}}
  >{\raggedright\arraybackslash}p{(\linewidth - 4\tabcolsep) * \real{0.3333}}
  >{\raggedright\arraybackslash}p{(\linewidth - 4\tabcolsep) * \real{0.3333}}@{}}
\toprule\noalign{}
\begin{minipage}[b]{\linewidth}\raggedright
\textbf{Feature}
\end{minipage} & \begin{minipage}[b]{\linewidth}\raggedright
\textbf{RSA (Asymmetric)}
\end{minipage} & \begin{minipage}[b]{\linewidth}\raggedright
\textbf{AES (Symmetric)}
\end{minipage} \\
\midrule\noalign{}
\endhead
\bottomrule\noalign{}
\endlastfoot
Speed & Slow & Very Fast \\
Suitable for & Small Data (Keys) & Large Data (Messages) \\
Security Level & Very High & Very High \\
\end{longtable}
}

If RSA were used to encrypt every chat message, communication would be
extremely slow.\\
If AES key were shared directly, an attacker could steal it.

So we \textbf{combine} them:

\begin{quote}
\textbf{RSA encrypts the AES session key}\\
\textbf{AES encrypts the messages}
\end{quote}

This approach is also used in \textbf{WhatsApp, Signal, Telegram,
SSL/TLS, and online banking}.

\subsection{\texorpdfstring{\textbf{4.2 System Generated Key
Pairs}}{4.2 System Generated Key Pairs}}\label{system-generated-key-pairs}

\subsubsection{\texorpdfstring{\textbf{On Server
Side}}{On Server Side}}\label{on-server-side}

The server generates an RSA key pair:

{\def\LTcaptype{none} % do not increment counter
\begin{longtable}[]{@{}
  >{\raggedright\arraybackslash}p{(\linewidth - 4\tabcolsep) * \real{0.1667}}
  >{\raggedright\arraybackslash}p{(\linewidth - 4\tabcolsep) * \real{0.2222}}
  >{\raggedright\arraybackslash}p{(\linewidth - 4\tabcolsep) * \real{0.4583}}@{}}
\toprule\noalign{}
\begin{minipage}[b]{\linewidth}\raggedright
\textbf{Key}
\end{minipage} & \begin{minipage}[b]{\linewidth}\raggedright
\textbf{Stored Where}
\end{minipage} & \begin{minipage}[b]{\linewidth}\raggedright
\textbf{Purpose}
\end{minipage} \\
\midrule\noalign{}
\endhead
\bottomrule\noalign{}
\endlastfoot
\textbf{Private Key} & Secure on server & Used to decrypt AES keys \\
\textbf{Public Key} & Shared with clients & Used by clients to encrypt
their AES key \\
\end{longtable}
}

\subsubsection{\texorpdfstring{\textbf{On Client Side
(Browser)}}{On Client Side (Browser)}}\label{on-client-side-browser}

The client generates:

{\def\LTcaptype{none} % do not increment counter
\begin{longtable}[]{@{}
  >{\raggedright\arraybackslash}p{(\linewidth - 4\tabcolsep) * \real{0.2055}}
  >{\raggedright\arraybackslash}p{(\linewidth - 4\tabcolsep) * \real{0.3836}}
  >{\raggedright\arraybackslash}p{(\linewidth - 4\tabcolsep) * \real{0.3836}}@{}}
\toprule\noalign{}
\begin{minipage}[b]{\linewidth}\raggedright
\textbf{Key}
\end{minipage} & \begin{minipage}[b]{\linewidth}\raggedright
\textbf{Visibility}
\end{minipage} & \begin{minipage}[b]{\linewidth}\raggedright
\textbf{Purpose}
\end{minipage} \\
\midrule\noalign{}
\endhead
\bottomrule\noalign{}
\endlastfoot
\textbf{AES Session Key} & Stored locally in browser memory & Encrypts
and decrypts messages \\
\end{longtable}
}

\subsection{\texorpdfstring{\textbf{4.3 Key Exchange Process
Overview}}{4.3 Key Exchange Process Overview}}\label{key-exchange-process-overview}

\textbf{Step 1:} Server generates RSA public/private key pair.

\textbf{Step 2:} Client requests server\textquotesingle s public key.

\textbf{Step 3:} Client generates its own AES-256 session key.

\textbf{Step 4:} Client encrypts AES key using server\textquotesingle s
RSA public key.

\textbf{Step 5:} Client sends encrypted AES key to server.

\textbf{Step 6:} Server decrypts AES key using private RSA key.

\textbf{Step 7:} Both server and client now share the same AES key.

\subsection{\texorpdfstring{\textbf{4.4 Key Exchange Flow
Diagram}}{4.4 Key Exchange Flow Diagram}}\label{key-exchange-flow-diagram}

\begin{figure}
\centering
\includegraphics[width=4.42153in,height=4.85764in,alt={ChatGPT Image Nov 10, 2025, 04\_17\_00 PM}]{/Users/sj/Desktop/devsprint/report-engine/workspace/32a2bd09-5eed-4bc4-8ba4-8d9fc92e7564/media/image2.png}
\caption{ChatGPT Image Nov 10, 2025, 04\_17\_00 PM}
\end{figure}

At this point:

\begin{itemize}
\item
  \textbf{Server never sees plaintext messages}
\item
  \textbf{Network attackers cannot derive the AES key}
\item
  \textbf{Session remains secure even on insecure networks}
\end{itemize}

\subsection{\texorpdfstring{\textbf{4.5 Security Advantages of This Key
Exchange}}{4.5 Security Advantages of This Key Exchange}}\label{security-advantages-of-this-key-exchange}

{\def\LTcaptype{none} % do not increment counter
\begin{longtable}[]{@{}
  >{\raggedright\arraybackslash}p{(\linewidth - 2\tabcolsep) * \real{0.3750}}
  >{\raggedright\arraybackslash}p{(\linewidth - 2\tabcolsep) * \real{0.5139}}@{}}
\toprule\noalign{}
\begin{minipage}[b]{\linewidth}\raggedright
\textbf{Security Requirement}
\end{minipage} & \begin{minipage}[b]{\linewidth}\raggedright
\textbf{How It Is Satisfied}
\end{minipage} \\
\midrule\noalign{}
\endhead
\bottomrule\noalign{}
\endlastfoot
\textbf{Confidentiality} & AES key is not exposed during transmission
because RSA encrypts it \\
\textbf{Integrity} & AES-GCM includes authentication tag preventing
message tampering \\
\textbf{Authentication} & Login + session keys tie identity to
encryption \\
\textbf{Forward Secrecy (Optional Extension)} & AES keys can be rotated
periodically \\
\end{longtable}
}

Even if attackers capture all messages, \textbf{they cannot decrypt
anything} without the AES session key.

\subsection{\texorpdfstring{\textbf{4.6 Key Storage and
Lifetime}}{4.6 Key Storage and Lifetime}}\label{key-storage-and-lifetime}

{\def\LTcaptype{none} % do not increment counter
\begin{longtable}[]{@{}
  >{\raggedright\arraybackslash}p{(\linewidth - 6\tabcolsep) * \real{0.2000}}
  >{\raggedright\arraybackslash}p{(\linewidth - 6\tabcolsep) * \real{0.3333}}
  >{\raggedright\arraybackslash}p{(\linewidth - 6\tabcolsep) * \real{0.2667}}
  >{\raggedright\arraybackslash}p{(\linewidth - 6\tabcolsep) * \real{0.1733}}@{}}
\toprule\noalign{}
\begin{minipage}[b]{\linewidth}\raggedright
\textbf{Key Type}
\end{minipage} & \begin{minipage}[b]{\linewidth}\raggedright
\textbf{Stored}
\end{minipage} & \begin{minipage}[b]{\linewidth}\raggedright
\textbf{Lifetime}
\end{minipage} & \begin{minipage}[b]{\linewidth}\raggedright
\textbf{Security Notes}
\end{minipage} \\
\midrule\noalign{}
\endhead
\bottomrule\noalign{}
\endlastfoot
RSA Private Key & On server (local file) & Long-term & Never shared \\
RSA Public Key & Sent to clients & Long-term & Safe to share \\
AES Session Key & In client memory + server session store & Active
session only & Deleted on logout \\
\end{longtable}
}

This ensures:

\begin{itemize}
\item
  No sensitive keys are hard-coded
\item
  Keys are not stored permanently
\item
  On logout AES key is destroyed communication becomes unreadable
\end{itemize}

\subsection{\texorpdfstring{\textbf{4.7 Real Example from Project
Code}}{4.7 Real Example from Project Code}}\label{real-example-from-project-code}

\subsubsection{Server Decrypts AES Key}\label{server-decrypts-aes-key}

aes\_key = rsa\_decrypt(encrypted\_key\_bytes)

\subsubsection{Client Encrypts AES Key Before
Sending}\label{client-encrypts-aes-key-before-sending}

const encryptedKey = await crypto.subtle.encrypt(\{name: "RSA-OAEP"\},
serverPublicKey, aesKeyRaw);

This confirms \textbf{RSA is used only once} at login making the system
fast.

\section{\texorpdfstring{\textbf{5. Encryption and Decryption
Process}}{5. Encryption and Decryption Process}}\label{encryption-and-decryption-process}

The primary goal of a secure messaging system is to ensure that messages
exchanged between users remain confidential and cannot be accessed or
manipulated by unauthorized entities. In this project, encryption and
decryption of messages are performed using the \textbf{AES-GCM symmetric
encryption algorithm}, while the AES key itself is protected via
\textbf{RSA asymmetric encryption}. This method ensures both
\textbf{security} and \textbf{performance}.

\subsection{\texorpdfstring{\textbf{5.1 Why AES is Used for Message
Encryption}}{5.1 Why AES is Used for Message Encryption}}\label{why-aes-is-used-for-message-encryption}

Once the client and server have securely exchanged the AES session key
using RSA, the actual chat messages are encrypted using \textbf{AES-GCM
(Advanced Encryption Standard -- Galois Counter Mode)}.

\subsubsection{\texorpdfstring{\textbf{Advantages of
AES-GCM}}{Advantages of AES-GCM}}\label{advantages-of-aes-gcm}

{\def\LTcaptype{none} % do not increment counter
\begin{longtable}[]{@{}
  >{\raggedright\arraybackslash}p{(\linewidth - 2\tabcolsep) * \real{0.4722}}
  >{\raggedright\arraybackslash}p{(\linewidth - 2\tabcolsep) * \real{0.3611}}@{}}
\toprule\noalign{}
\begin{minipage}[b]{\linewidth}\raggedright
\textbf{Feature}
\end{minipage} & \begin{minipage}[b]{\linewidth}\raggedright
\textbf{Benefit}
\end{minipage} \\
\midrule\noalign{}
\endhead
\bottomrule\noalign{}
\endlastfoot
Fast encryption/decryption & Suitable for real-time chat \\
Strong security (256-bit key) & Resistant to brute-force attacks \\
Includes authentication tag & Detects message manipulation \\
Works efficiently in browsers and servers & Supported by WebCrypto
API \\
\end{longtable}
}

Thus, \textbf{AES encrypts the message before sending} and
\textbf{decrypts it on the receiver side}.

\subsection{\texorpdfstring{\textbf{5.2 Encryption Process (Sender
Side)}}{5.2 Encryption Process (Sender Side)}}\label{encryption-process-sender-side}

When the sender types a message:

\begin{itemize}
\item
  The message text is converted to bytes.
\item
  A unique \textbf{Initialization Vector (IV)} (12 bytes) is generated.
\item
  AES-GCM encrypts the plaintext using:

  \begin{itemize}
  \item
    AES session key
  \item
    IV
  \end{itemize}
\item
  The encrypted output (ciphertext) is converted to Base64 for network
  transmission.
\item
  The IV and ciphertext are sent to the server then forwarded to the
  intended recipient.
\end{itemize}

\subsubsection{\texorpdfstring{\textbf{Encryption
Flow}}{Encryption Flow}}\label{encryption-flow}

\begin{figure}
\centering
\includegraphics[width=5.05486in,height=3.86319in,alt={ChatGPT Image Nov 10, 2025, 04\_27\_33 PM}]{/Users/sj/Desktop/devsprint/report-engine/workspace/32a2bd09-5eed-4bc4-8ba4-8d9fc92e7564/media/image3.png}
\caption{ChatGPT Image Nov 10, 2025, 04\_27\_33 PM}
\end{figure}

\subsection{\texorpdfstring{\textbf{5.3 Decryption Process (Recipient
Side)}}{5.3 Decryption Process (Recipient Side)}}\label{decryption-process-recipient-side}

\begin{itemize}
\item
  When the recipient receives the ciphertext:
\item
  The Base64 ciphertext is converted back into bytes.
\item
  AES-GCM decrypts it using:

  \begin{itemize}
  \item
    The AES session key
  \item
    The IV sent alongside the message
  \end{itemize}
\item
  The result is plaintext.
\item
  Plaintext is displayed to the user.
\end{itemize}

\subsubsection{\texorpdfstring{\textbf{Decryption
Flow}}{Decryption Flow}}\label{decryption-flow}

\subsubsection{\texorpdfstring{\protect\includegraphics[width=4.40556in,height=3.69028in,alt={ChatGPT Image Nov 10, 2025, 04\_31\_01 PM}]{/Users/sj/Desktop/devsprint/report-engine/workspace/32a2bd09-5eed-4bc4-8ba4-8d9fc92e7564/media/image4.png}}{ChatGPT Image Nov 10, 2025, 04\_31\_01 PM}}\label{chatgpt-image-nov-10-2025-04_31_01-pm}

\subsection{\texorpdfstring{\textbf{5.4 Worked Example
(Step-by-Step)}}{5.4 Worked Example (Step-by-Step)}}\label{worked-example-step-by-step}

Assume:

\begin{itemize}
\item
  AES key (shared session key):\\
  K = "2f 7f 9e 34 12 ab 45 ce ..." (256-bit key)
\item
  Plaintext message:\\
  "HELLO"
\end{itemize}

\subsubsection{\texorpdfstring{\textbf{Step 1: Convert to
Bytes}}{Step 1: Convert to Bytes}}\label{step-1-convert-to-bytes}

"HELLO" 48 45 4C 4C 4F (in hex)

\subsubsection{\texorpdfstring{\textbf{Step 2: Generate Random
IV}}{Step 2: Generate Random IV}}\label{step-2-generate-random-iv}

Example IV:

IV = 9a 5c 11 3f 52 e0 8f c4 97 21 aa 3d (12 bytes)

\subsubsection{\texorpdfstring{\textbf{Step 3: AES-GCM
Encryption}}{Step 3: AES-GCM Encryption}}\label{step-3-aes-gcm-encryption}

AES encrypts plaintext into ciphertext:

Ciphertext = d4 8a 91 e1 71 56 2d cb 8f 33

\subsubsection{\texorpdfstring{\textbf{Step 4: Convert to Base64 for
Transmission}}{Step 4: Convert to Base64 for Transmission}}\label{step-4-convert-to-base64-for-transmission}

IV "mlwRP1Lgj8SXIao9"

Ciphertext "1IqR4XFWLcuPMw=="

The server receives:

\{

"iv\_b64": "mlwRP1Lgj8SXIao9",

"ciphertext\_b64": "1IqR4XFWLcuPMw=="

\}

\subsection{\texorpdfstring{\textbf{5.5 Decryption Example (Recipient
Side)}}{5.5 Decryption Example (Recipient Side)}}\label{decryption-example-recipient-side}

Recipient receives IV + ciphertext uses same AES key to decrypt:

AES\_Decrypt( K, IV, Ciphertext ) Plaintext = "HELLO"

Since AES-GCM checks \textbf{integrity}, if the ciphertext was modified:

Decryption Fails Message is rejected

Thus ensuring protection against \textbf{tampering attacks}.

\subsection{\texorpdfstring{\textbf{5.6 Security Advantages of This
Encryption
Scheme}}{5.6 Security Advantages of This Encryption Scheme}}\label{security-advantages-of-this-encryption-scheme}

{\def\LTcaptype{none} % do not increment counter
\begin{longtable}[]{@{}
  >{\raggedright\arraybackslash}p{(\linewidth - 4\tabcolsep) * \real{0.2500}}
  >{\raggedright\arraybackslash}p{(\linewidth - 4\tabcolsep) * \real{0.3611}}
  >{\raggedright\arraybackslash}p{(\linewidth - 4\tabcolsep) * \real{0.3611}}@{}}
\toprule\noalign{}
\begin{minipage}[b]{\linewidth}\raggedright
\textbf{Security Feature}
\end{minipage} & \begin{minipage}[b]{\linewidth}\raggedright
\textbf{Provided By}
\end{minipage} & \begin{minipage}[b]{\linewidth}\raggedright
\textbf{Result}
\end{minipage} \\
\midrule\noalign{}
\endhead
\bottomrule\noalign{}
\endlastfoot
Confidentiality & AES-256 Encryption & Prevents eavesdropping \\
Integrity & AES-GCM Authentication Tag & Detects modification \\
Key Protection & RSA-OAEP Key Encryption & Prevents key theft \\
Replay Protection & New IV for every message & Prevents old message
injection \\
\end{longtable}
}

This ensures the system is secure even over an \textbf{insecure network
like public WiFi}.

\section{\texorpdfstring{\textbf{6. Real-Time Messaging Using
WebSockets}}{6. Real-Time Messaging Using WebSockets}}\label{real-time-messaging-using-websockets}

Real-time communication is an essential requirement in modern messaging
applications. Traditional HTTP communication follows a
\textbf{request--response model}, where the client must repeatedly
request new data from the server. This creates significant delay,
increased bandwidth usage, and poor user experience. To overcome these
limitations, the Secure Messaging System uses \textbf{WebSockets}, which
allow persistent, bidirectional communication between client and server.

\subsection{\texorpdfstring{\textbf{6.1 Limitations of HTTP
Polling}}{6.1 Limitations of HTTP Polling}}\label{limitations-of-http-polling}

Before WebSockets, chat applications relied on techniques such as:

{\def\LTcaptype{none} % do not increment counter
\begin{longtable}[]{@{}
  >{\raggedright\arraybackslash}p{(\linewidth - 4\tabcolsep) * \real{0.2083}}
  >{\raggedright\arraybackslash}p{(\linewidth - 4\tabcolsep) * \real{0.4306}}
  >{\raggedright\arraybackslash}p{(\linewidth - 4\tabcolsep) * \real{0.2917}}@{}}
\toprule\noalign{}
\begin{minipage}[b]{\linewidth}\raggedright
\textbf{Method}
\end{minipage} & \begin{minipage}[b]{\linewidth}\raggedright
\textbf{Description}
\end{minipage} & \begin{minipage}[b]{\linewidth}\raggedright
\textbf{Drawbacks}
\end{minipage} \\
\midrule\noalign{}
\endhead
\bottomrule\noalign{}
\endlastfoot
\textbf{Polling} & Client sends requests periodically to check for new
messages & Wastes bandwidth, delays delivery \\
\textbf{Long Polling} & Client waits until server responds with new data
& Reduces delay but still inefficient \\
\textbf{Refresh / Reload} & Manual page refresh to get updates & Poor
user experience \\
\end{longtable}
}

These methods are either \textbf{inefficient} or \textbf{slow},
especially when real-time responsiveness is required.

\subsection{\texorpdfstring{\textbf{6.2 Introduction to
WebSockets}}{6.2 Introduction to WebSockets}}\label{introduction-to-websockets}

WebSockets provide a \textbf{full-duplex (two-way)} communication
channel over a \textbf{single, long-lived TCP connection}. Once the
WebSocket connection is established, both the server and the client can
send or receive data at any time \textbf{without needing repeated
requests}.

\subsubsection{\texorpdfstring{\textbf{Key Properties of
WebSockets}}{Key Properties of WebSockets}}\label{key-properties-of-websockets}

{\def\LTcaptype{none} % do not increment counter
\begin{longtable}[]{@{}
  >{\raggedright\arraybackslash}p{(\linewidth - 2\tabcolsep) * \real{0.3425}}
  >{\raggedright\arraybackslash}p{(\linewidth - 2\tabcolsep) * \real{0.6301}}@{}}
\toprule\noalign{}
\begin{minipage}[b]{\linewidth}\raggedright
\textbf{Feature}
\end{minipage} & \begin{minipage}[b]{\linewidth}\raggedright
\textbf{Benefit}
\end{minipage} \\
\midrule\noalign{}
\endhead
\bottomrule\noalign{}
\endlastfoot
Persistent connection & No repeated handshake or overhead \\
Bidirectional communication & Both client and server can send messages
independently \\
Low-latency and real-time & Messages are delivered instantly \\
Efficient resource usage & No repeated HTTP requests \\
\end{longtable}
}

\subsection{\texorpdfstring{\textbf{6.3 WebSocket Workflow in This
Project}}{6.3 WebSocket Workflow in This Project}}\label{websocket-workflow-in-this-project}

Client Logs In

Client opens WebSocket connection to server

\textbf{\emph{ws://server-address/ws/\{username\}}}

Server registers the WebSocket connection for the user

When a message arrives for a user:

\textbf{\emph{Server sends encrypted message over WebSocket instantly}}

Client receives encrypted message

Client decrypts using AES session key

Message is displayed to user in chat window

This ensures the message is delivered in \textbf{real-time} without
polling.

\subsection{\texorpdfstring{\textbf{6.4 WebSocket Connection
Diagram}}{6.4 WebSocket Connection Diagram}}\label{websocket-connection-diagram}

\begin{figure}
\centering
\includegraphics[width=4.75in,height=3.99722in,alt={ChatGPT Image Nov 10, 2025, 04\_41\_43 PM}]{/Users/sj/Desktop/devsprint/report-engine/workspace/32a2bd09-5eed-4bc4-8ba4-8d9fc92e7564/media/image5.png}
\caption{ChatGPT Image Nov 10, 2025, 04\_41\_43 PM}
\end{figure}

\subsection{\texorpdfstring{\textbf{6.5 WebSocket Subscription per
User}}{6.5 WebSocket Subscription per User}}\label{websocket-subscription-per-user}

Each user maintains \textbf{one active WebSocket connection}.

@app.websocket("/ws/\{username\}")async def
websocket\_endpoint(websocket: WebSocket, username: str):

await manager.connect(username, websocket)

try:

while True:

await websocket.receive\_text() \# keeps the connection alive

except WebSocketDisconnect:

manager.disconnect(username, websocket)

The server uses a \textbf{Connection Manager} to track which user is
online.

\subsection{\texorpdfstring{\textbf{6.6 Sending Messages Over WebSocket
(Encrypted)}}{6.6 Sending Messages Over WebSocket (Encrypted)}}\label{sending-messages-over-websocket-encrypted}

Once the server processes and re-encrypts a message for the recipient,
it pushes it:

await manager.send\_personal(

recipient\_username,

\{

"from": sender\_username,

"iv\_b64": base64\_iv,

"ciphertext\_b64": base64\_cipher,

"timestamp": ts

\}

)

This ensures:

\begin{itemize}
\item
  No plaintext ever travels over the network
\item
  Only authenticated, logged-in users can receive messages
\end{itemize}

\subsection{\texorpdfstring{\textbf{6.7 Receiving and Decrypting on
Client}}{6.7 Receiving and Decrypting on Client}}\label{receiving-and-decrypting-on-client}

On the browser side:

ws.onmessage = async (event) =\textgreater{} \{

const data = JSON.parse(event.data);

const iv = base64ToBytes(data.iv\_b64);

const ct = base64ToBytes(data.ciphertext\_b64);

const pt = await aesDecryptRaw(aesKeyObj, iv, ct);

const messageText = new TextDecoder().decode(pt);

setMessages(prev =\textgreater{} {[}...prev, \{ from: data.from, text:
messageText \}{]});

\};

The message is decrypted and displayed \textbf{instantly}.

\subsection{\texorpdfstring{\textbf{6.8 Advantages of Using WebSockets
in Secure
Messaging}}{6.8 Advantages of Using WebSockets in Secure Messaging}}\label{advantages-of-using-websockets-in-secure-messaging}

{\def\LTcaptype{none} % do not increment counter
\begin{longtable}[]{@{}
  >{\raggedright\arraybackslash}p{(\linewidth - 2\tabcolsep) * \real{0.3056}}
  >{\raggedright\arraybackslash}p{(\linewidth - 2\tabcolsep) * \real{0.4722}}@{}}
\toprule\noalign{}
\begin{minipage}[b]{\linewidth}\raggedright
\textbf{Feature}
\end{minipage} & \begin{minipage}[b]{\linewidth}\raggedright
\textbf{Impact}
\end{minipage} \\
\midrule\noalign{}
\endhead
\bottomrule\noalign{}
\endlastfoot
Real-time updates & Messages appear instantly without delay \\
Efficient bandwidth usage & No repeated HTTP requests \\
Better scalability & Server handles fewer requests \\
Improved user experience & Smooth, responsive chat interface \\
\end{longtable}
}

\section{\texorpdfstring{\textbf{7. Intrusion Detection System
(IDS)}}{7. Intrusion Detection System (IDS)}}\label{intrusion-detection-system-ids-1}

As digital communication systems expand, the risk of unauthorized
access, brute-force attacks, and malicious intrusion attempts increases
significantly. To safeguard user accounts and ensure secure
communication, the Secure Messaging System integrates a
\textbf{lightweight Intrusion Detection System (IDS)} that continuously
monitors login activity and detects suspicious behavior in real-time.

The IDS does not interfere with regular messaging functionality but
enhances system security by identifying and blocking potentially harmful
access attempts before they can compromise user privacy.

\subsection{\texorpdfstring{\textbf{7.1 Need for Intrusion Detection in
Secure
Messaging}}{7.1 Need for Intrusion Detection in Secure Messaging}}\label{need-for-intrusion-detection-in-secure-messaging}

Even when cryptographic encryption protects message content, systems can
still be vulnerable to attacks such as:

{\def\LTcaptype{none} % do not increment counter
\begin{longtable}[]{@{}
  >{\raggedright\arraybackslash}p{(\linewidth - 4\tabcolsep) * \real{0.2639}}
  >{\raggedright\arraybackslash}p{(\linewidth - 4\tabcolsep) * \real{0.4444}}
  >{\raggedright\arraybackslash}p{(\linewidth - 4\tabcolsep) * \real{0.2639}}@{}}
\toprule\noalign{}
\begin{minipage}[b]{\linewidth}\raggedright
\textbf{Attack Type}
\end{minipage} & \begin{minipage}[b]{\linewidth}\raggedright
\textbf{Description}
\end{minipage} & \begin{minipage}[b]{\linewidth}\raggedright
\textbf{Risk}
\end{minipage} \\
\midrule\noalign{}
\endhead
\bottomrule\noalign{}
\endlastfoot
\textbf{Brute Force Attack} & Attacker repeatedly attempts login with
many password guesses & Unauthorized account access \\
\textbf{Credential Stuffing} & Using leaked passwords from other sites &
Account takeover \\
\textbf{Account Enumeration} & Detecting if usernames exist & Privacy
leakage \\
\textbf{Session Hijacking} & Attacker takes over active sessions &
Impersonation \\
\end{longtable}
}

Therefore, detecting \textbf{patterns of misuse} is essential to ensure
system security.

\subsection{\texorpdfstring{\textbf{7.2 IDS Strategy Used in This
Project}}{7.2 IDS Strategy Used in This Project}}\label{ids-strategy-used-in-this-project}

The IDS implemented in this system focuses on \textbf{Login Attempt
Monitoring}. It tracks:

\begin{itemize}
\item
  The \textbf{IP Address} from which login attempts originate.
\item
  The \textbf{number of failed login attempts} in a short time period.
\end{itemize}

If suspicious behavior is detected, the system \textbf{temporarily
blocks} that IP address.

\subsubsection{Key Principle:}\label{key-principle}

\begin{quote}
If a user fails to authenticate repeatedly in a short time window, it
indicates possible \textbf{attack activity}.
\end{quote}

\subsection{\texorpdfstring{\textbf{7.3 IDS Working
Logic}}{7.3 IDS Working Logic}}\label{ids-working-logic}

The IDS maintains two in-memory structures:

{\def\LTcaptype{none} % do not increment counter
\begin{longtable}[]{@{}
  >{\raggedright\arraybackslash}p{(\linewidth - 2\tabcolsep) * \real{0.3056}}
  >{\centering\arraybackslash}p{(\linewidth - 2\tabcolsep) * \real{0.5556}}@{}}
\toprule\noalign{}
\begin{minipage}[b]{\linewidth}\raggedright
\textbf{Variable}
\end{minipage} & \begin{minipage}[b]{\linewidth}\centering
\textbf{Description}
\end{minipage} \\
\midrule\noalign{}
\endhead
\bottomrule\noalign{}
\endlastfoot
failed\_logins\_by\_ip & Maps IP timestamps of failed login attempts \\
blocked\_ips & Maps IP time until the block expires \\
\end{longtable}
}

\subsubsection{Threshold Parameters:}\label{threshold-parameters}

{\def\LTcaptype{none} % do not increment counter
\begin{longtable}[]{@{}
  >{\raggedright\arraybackslash}p{(\linewidth - 4\tabcolsep) * \real{0.2917}}
  >{\raggedright\arraybackslash}p{(\linewidth - 4\tabcolsep) * \real{0.2083}}
  >{\raggedright\arraybackslash}p{(\linewidth - 4\tabcolsep) * \real{0.3472}}@{}}
\toprule\noalign{}
\begin{minipage}[b]{\linewidth}\raggedright
\textbf{Parameter}
\end{minipage} & \begin{minipage}[b]{\linewidth}\raggedright
\textbf{Value}
\end{minipage} & \begin{minipage}[b]{\linewidth}\raggedright
\textbf{Meaning}
\end{minipage} \\
\midrule\noalign{}
\endhead
\bottomrule\noalign{}
\endlastfoot
BLOCK\_THRESHOLD & 5 failures & Max allowed failed attempts \\
BLOCK\_WINDOW & 300 seconds & Time window for tracking attempts \\
\end{longtable}
}

\subsection{}\label{section-9}

\subsection{\texorpdfstring{\textbf{7.4 IDS Algorithm
(Step-by-Step)}}{7.4 IDS Algorithm (Step-by-Step)}}\label{ids-algorithm-step-by-step}

For every login attempt:

1. Extract client IP address.

2. Check if IP is currently blocked:

If yes deny login request.

3. If login is incorrect:

Add timestamp to failed\_logins\_by\_ip list for that IP.

Remove old timestamps older than BLOCK\_WINDOW.

If number of failures BLOCK\_THRESHOLD:

Add IP to blocked\_ips with unblock\_time = now + BLOCK\_WINDOW.

4. If login is correct:

Clear failed login count for that IP.

Allow login and issue JWT.

\subsection{\texorpdfstring{\textbf{7.5 IDS Process Flow
Diagram}}{7.5 IDS Process Flow Diagram}}\label{ids-process-flow-diagram}

\begin{figure}
\centering
\includegraphics[width=4.11528in,height=4.54236in,alt={ChatGPT Image Nov 10, 2025, 05\_50\_48 PM}]{/Users/sj/Desktop/devsprint/report-engine/workspace/32a2bd09-5eed-4bc4-8ba4-8d9fc92e7564/media/image6.png}
\caption{ChatGPT Image Nov 10, 2025, 05\_50\_48 PM}
\end{figure}

\subsection{\texorpdfstring{\textbf{7.6 Example
Scenario}}{7.6 Example Scenario}}\label{example-scenario}

{\def\LTcaptype{none} % do not increment counter
\begin{longtable}[]{@{}
  >{\raggedright\arraybackslash}p{(\linewidth - 6\tabcolsep) * \real{0.1528}}
  >{\raggedright\arraybackslash}p{(\linewidth - 6\tabcolsep) * \real{0.1944}}
  >{\raggedright\arraybackslash}p{(\linewidth - 6\tabcolsep) * \real{0.2083}}
  >{\raggedright\arraybackslash}p{(\linewidth - 6\tabcolsep) * \real{0.3333}}@{}}
\toprule\noalign{}
\begin{minipage}[b]{\linewidth}\raggedright
\textbf{Time}
\end{minipage} & \begin{minipage}[b]{\linewidth}\raggedright
\textbf{IP Address}
\end{minipage} & \begin{minipage}[b]{\linewidth}\raggedright
\textbf{Login Outcome}
\end{minipage} & \begin{minipage}[b]{\linewidth}\raggedright
\textbf{IDS Action}
\end{minipage} \\
\midrule\noalign{}
\endhead
\bottomrule\noalign{}
\endlastfoot
10:30:01 & 192.168.1.4 & Wrong Password & Count = 1 \\
10:30:08 & 192.168.1.4 & Wrong Password & Count = 2 \\
10:30:16 & 192.168.1.4 & Wrong Password & Count = 3 \\
10:30:29 & 192.168.1.4 & Wrong Password & Count = 4 \\
10:30:41 & 192.168.1.4 & Wrong Password & \textbf{Count = 5 IP
BLOCKED} \\
\end{longtable}
}

If the same IP tries to login again within the block window:

Response: "403 -- Access Blocked due to Suspicious Activity"

\subsection{\texorpdfstring{\textbf{7.7 IDS Integration in the
System}}{7.7 IDS Integration in the System}}\label{ids-integration-in-the-system}

The IDS is integrated directly into the /login authentication route of
the backend server.

This ensures:

\begin{itemize}
\item
  No additional hardware/software is required.
\item
  Intrusion detection occurs \textbf{before} a session token is issued.
\item
  Attacks are \textbf{prevented early}, without affecting message
  encryption logic.
\end{itemize}

\subsection{\texorpdfstring{\textbf{7.8 Strengths and
Limitations}}{7.8 Strengths and Limitations}}\label{strengths-and-limitations}

{\def\LTcaptype{none} % do not increment counter
\begin{longtable}[]{@{}
  >{\raggedright\arraybackslash}p{(\linewidth - 2\tabcolsep) * \real{0.4861}}
  >{\raggedright\arraybackslash}p{(\linewidth - 2\tabcolsep) * \real{0.4861}}@{}}
\toprule\noalign{}
\begin{minipage}[b]{\linewidth}\raggedright
\textbf{Strengths}
\end{minipage} & \begin{minipage}[b]{\linewidth}\raggedright
\textbf{Limitations}
\end{minipage} \\
\midrule\noalign{}
\endhead
\bottomrule\noalign{}
\endlastfoot
Simple and efficient & Does not detect advanced attacks like
anomaly-based intrusion \\
Adds strong protection against brute force & Stores IP data only in
memory \\
No performance overhead & IP spoofing may bypass tracking \\
Ideal for light/medium traffic systems & Needs extension for
enterprise-level environments \\
\end{longtable}
}

\subsection{\texorpdfstring{\textbf{7.9 Possible
Enhancements}}{7.9 Possible Enhancements}}\label{possible-enhancements}

To make the IDS more robust:

\begin{itemize}
\item
  Store IDS logs in database for persistence.
\item
  Use Machine Learning to detect abnormal login patterns.
\item
  Integrate \textbf{Snort / Suricata} network intrusion monitoring.
\item
  Add email/SMS alerts for administrators.
\end{itemize}

\section{}\label{section-10}

\section{}\label{section-11}

\section{}\label{section-12}

\section{}\label{section-13}

\section{}\label{section-14}

\section{}\label{section-15}

\section{}\label{section-16}

\section{}\label{section-17}

\section{}\label{section-18}

\section{}\label{section-19}

\section{}\label{section-20}

\section{\texorpdfstring{\textbf{8. Implementation
Details}}{8. Implementation Details}}\label{implementation-details}

The Secure Messaging System is implemented using a
\textbf{client--server architecture} with a \textbf{React-based
frontend} and a \textbf{FastAPI backend}. The system integrates
cryptographic modules, real-time message delivery services,
authentication services, and an intrusion detection component.

This section explains the implementation structure, module
responsibilities, and workflow at the code level.

\subsection{\texorpdfstring{\textbf{8.1 System Modules
Overview}}{8.1 System Modules Overview}}\label{system-modules-overview}

{\def\LTcaptype{none} % do not increment counter
\begin{longtable}[]{@{}
  >{\raggedright\arraybackslash}p{(\linewidth - 4\tabcolsep) * \real{0.2917}}
  >{\raggedright\arraybackslash}p{(\linewidth - 4\tabcolsep) * \real{0.2500}}
  >{\raggedright\arraybackslash}p{(\linewidth - 4\tabcolsep) * \real{0.3333}}@{}}
\toprule\noalign{}
\begin{minipage}[b]{\linewidth}\raggedright
\textbf{Module Name}
\end{minipage} & \begin{minipage}[b]{\linewidth}\raggedright
\textbf{Location}
\end{minipage} & \begin{minipage}[b]{\linewidth}\raggedright
\textbf{Purpose}
\end{minipage} \\
\midrule\noalign{}
\endhead
\bottomrule\noalign{}
\endlastfoot
\textbf{Authentication Module} & Backend (FastAPI) & Handles user signup
and login using hashed credentials \\
\textbf{Key Management Module} & Backend + Frontend & Performs RSA
public-key exchange and AES session key setup \\
\textbf{Encryption / Decryption Module} & Frontend \& Backend & Encrypts
messages using AES-GCM and re-encrypts for recipients \\
\textbf{Messaging Module} & WebSocket \& HTTP & Sends and receives
encrypted messages in real-time \\
\textbf{Intrusion Detection Module (IDS)} & Backend & Monitors login
activity and blocks suspicious attempts \\
\textbf{User Interface Module} & React (Tailwind UI) & Provides chat
interface and live message updates \\
\end{longtable}
}

\subsection{\texorpdfstring{\textbf{8.2 Backend Implementation
(FastAPI)}}{8.2 Backend Implementation (FastAPI)}}\label{backend-implementation-fastapi}

\subsubsection{\texorpdfstring{\textbf{8.2.1 Project Structure
(Backend)}}{8.2.1 Project Structure (Backend)}}\label{project-structure-backend}

backend/

main.py \# Main API Server

crypto\_utils.py \# RSA + AES encryption utilities

ws\_manager.py \# WebSocket connection manager

chat.db \# SQLite database

server\_public\_key.pem \# RSA public key

server\_private\_key.pem \# RSA private key

\subsubsection{\texorpdfstring{\textbf{8.2.2 User
Authentication}}{8.2.2 User Authentication}}\label{user-authentication}

Passwords are \textbf{never stored in plaintext}.\\
They are hashed using \textbf{bcrypt} and stored in the database.

\textbf{Key functions:}

hashed\_password = bcrypt.hashpw(password.encode(), bcrypt.gensalt())

bcrypt.checkpw(input\_password.encode(), stored\_password\_hash)

A valid login returns a \textbf{JWT token} used for authorization in all
requests.

\subsubsection{\texorpdfstring{\textbf{8.2.3 RSA Key
Management}}{8.2.3 RSA Key Management}}\label{rsa-key-management}

On server startup:

generate\_rsa\_keys()

\begin{itemize}
\item
  The \textbf{public key} is returned to clients via /public\_key.
\item
  The \textbf{private key} never leaves the server.
\end{itemize}

\subsubsection{\texorpdfstring{\textbf{8.2.4 AES Session Key
Exchange}}{8.2.4 AES Session Key Exchange}}\label{aes-session-key-exchange}

Client generates AES key encrypts with RSA public key sends to server:

@app.post("/session\_key")def session\_key():

aes\_key = rsa\_decrypt(encrypted\_key)

session\_keys{[}username{]} = aes\_key

This ensures:\\
Key confidentiality\\
No insecure key transmission

\subsubsection{\texorpdfstring{\textbf{8.2.5 Message Send \&
Re-Encryption
Logic}}{8.2.5 Message Send \& Re-Encryption Logic}}\label{message-send-re-encryption-logic}

When a sender sends a message:

\begin{enumerate}
\def\labelenumi{\arabic{enumi}.}
\item
  Sender encrypts message with AES.
\item
  Server decrypts and obtains plaintext.
\item
  Server re-encrypts plaintext \textbf{with recipient\textquotesingle s
  AES key}.
\item
  Server stores + forwards encrypted message.
\end{enumerate}

plaintext = aes\_decrypt(sender\_key, iv, ciphertext)

rcipher = aes\_encrypt(recipient\_key, new\_iv, plaintext)

This provides:

\begin{itemize}
\item
  End-to-end style encryption
\item
  Server cannot modify message without detection
\end{itemize}

\subsubsection{\texorpdfstring{\textbf{8.2.6 WebSocket Real-Time
Delivery}}{8.2.6 WebSocket Real-Time Delivery}}\label{websocket-real-time-delivery}

The server uses a \textbf{connection manager} to track who is online.

await manager.send\_personal(recipient, encrypted\_message\_payload)

Thus, messages appear \textbf{instantly}.

\subsubsection{\texorpdfstring{\textbf{8.2.7 Intrusion Detection
(IDS)}}{8.2.7 Intrusion Detection (IDS)}}\label{intrusion-detection-ids}

IDS monitors \textbf{failed login attempts} from each IP.

If failures threshold:

blocked\_ips{[}ip{]} = timestamp + BLOCK\_WINDOW

This prevents:

\begin{itemize}
\item
  Brute-force attacks
\item
  Credential stuffing attempts
\end{itemize}

\subsection{\texorpdfstring{\textbf{8.3 Frontend Implementation (React +
Tailwind)}}{8.3 Frontend Implementation (React + Tailwind)}}\label{frontend-implementation-react-tailwind}

\subsubsection{\texorpdfstring{\textbf{8.3.1 Project Structure
(Frontend)}}{8.3.1 Project Structure (Frontend)}}\label{project-structure-frontend}

frontend/

src/

App.jsx

Login.jsx

Chat.jsx

api.js \# Backend API calls

cryptoClient.js \# AES/RSA/WebCrypto operations

tailwind.config.js \# UI theme

\subsubsection{\texorpdfstring{\textbf{8.3.2 Login
Workflow}}{8.3.2 Login Workflow}}\label{login-workflow}

\begin{itemize}
\item
  User enters credentials.
\item
  Password sent securely Backend verifies.
\item
  Backend returns JWT.
\item
  Frontend saves token and continues.
\end{itemize}

\subsubsection{\texorpdfstring{\textbf{8.3.3 AES Key Generation and
Storage}}{8.3.3 AES Key Generation and Storage}}\label{aes-key-generation-and-storage}

AES key is generated \textbf{locally} using Web Crypto API:

const key = await crypto.subtle.generateKey(\{name:"AES-GCM",
length:256\}, true, {[}"encrypt","decrypt"{]});

The raw key is exported and encrypted using RSA before sending to
server.

\subsubsection{\texorpdfstring{\textbf{8.3.4 WebSocket Client
Logic}}{8.3.4 WebSocket Client Logic}}\label{websocket-client-logic}

const ws = new WebSocket(`ws://localhost:8000/ws/\$\{username\}`);

ws.onmessage = async (event) =\textgreater{} \{

decrypt display message

\}

No polling \textbf{real-time UI updates}.

\subsubsection{\texorpdfstring{\textbf{8.3.5 Message Encryption in
Client}}{8.3.5 Message Encryption in Client}}\label{message-encryption-in-client}

const \{iv, ct\} = await aesEncryptRaw(aesKeyObj,
plaintextBytes);sendMessage(token, {[}recipient{]}, base64(iv),
base64(ct));

Recipient decrypts using \textbf{same AES session key}.

\subsection{\texorpdfstring{\textbf{8.4 Database
Schema}}{8.4 Database Schema}}\label{database-schema}

{\def\LTcaptype{none} % do not increment counter
\begin{longtable}[]{@{}
  >{\raggedright\arraybackslash}p{(\linewidth - 4\tabcolsep) * \real{0.1507}}
  >{\raggedright\arraybackslash}p{(\linewidth - 4\tabcolsep) * \real{0.4658}}
  >{\raggedright\arraybackslash}p{(\linewidth - 4\tabcolsep) * \real{0.3562}}@{}}
\toprule\noalign{}
\begin{minipage}[b]{\linewidth}\raggedright
\textbf{Table Name}
\end{minipage} & \begin{minipage}[b]{\linewidth}\raggedright
\textbf{Columns}
\end{minipage} & \begin{minipage}[b]{\linewidth}\raggedright
\textbf{Purpose}
\end{minipage} \\
\midrule\noalign{}
\endhead
\bottomrule\noalign{}
\endlastfoot
users & id, username, password\_hash & User identity storage \\
messages & id, sender, recipient, iv, ciphertext, timestamp & Stores
encrypted offline messages \\
\end{longtable}
}

All messages are \textbf{stored encrypted}, not as plaintext.

\subsection{\texorpdfstring{\textbf{8.5 Summary of
Implementation}}{8.5 Summary of Implementation}}\label{summary-of-implementation}

{\def\LTcaptype{none} % do not increment counter
\begin{longtable}[]{@{}
  >{\raggedright\arraybackslash}p{(\linewidth - 2\tabcolsep) * \real{0.2917}}
  >{\raggedright\arraybackslash}p{(\linewidth - 2\tabcolsep) * \real{0.3194}}@{}}
\toprule\noalign{}
\begin{minipage}[b]{\linewidth}\raggedright
\textbf{Security Goal}
\end{minipage} & \begin{minipage}[b]{\linewidth}\raggedright
\textbf{Achieved By}
\end{minipage} \\
\midrule\noalign{}
\endhead
\bottomrule\noalign{}
\endlastfoot
Confidentiality & AES Encryption \\
Authentication & JWT + bcrypt \\
Integrity & AES-GCM auth tag \\
Secure Key Sharing & RSA-OAEP encryption \\
Attack Prevention & Built-in IDS monitoring \\
\end{longtable}
}

The implementation \textbf{aligns with real-world secure messaging
systems} such as WhatsApp, Signal, and Telegram.

\section{}\label{section-21}

\section{}\label{section-22}

\section{}\label{section-23}

\section{}\label{section-24}

\section{}\label{section-25}

\section{}\label{section-26}

\section{}\label{section-27}

\section{}\label{section-28}

\section{}\label{section-29}

\section{\texorpdfstring{\textbf{9. Testing, Output Screenshots, and
Result
Analysis}}{9. Testing, Output Screenshots, and Result Analysis}}\label{testing-output-screenshots-and-result-analysis}

The Secure Messaging System was thoroughly tested to ensure correct
functionality, cryptographic security, real-time responsiveness, and
intrusion detection accuracy. Testing was performed on multiple devices
and browsers to validate the user experience and communication
reliability.

Testing was conducted in the following environment:

{\def\LTcaptype{none} % do not increment counter
\begin{longtable}[]{@{}
  >{\raggedright\arraybackslash}p{(\linewidth - 2\tabcolsep) * \real{0.2222}}
  >{\raggedright\arraybackslash}p{(\linewidth - 2\tabcolsep) * \real{0.6944}}@{}}
\toprule\noalign{}
\begin{minipage}[b]{\linewidth}\raggedright
\textbf{Component}
\end{minipage} & \begin{minipage}[b]{\linewidth}\raggedright
\textbf{Specification}
\end{minipage} \\
\midrule\noalign{}
\endhead
\bottomrule\noalign{}
\endlastfoot
Operating System & Windows / Ubuntu \\
Frontend Runtime & Google Chrome Browser \\
Backend Runtime & Python 3.10+, FastAPI, Uvicorn \\
Database & SQLite \\
Tools Used & Browser DevTools, Wireshark (optional), Postman
(optional) \\
\end{longtable}
}

\subsection{\texorpdfstring{\textbf{9.1 Test Cases and
Results}}{9.1 Test Cases and Results}}\label{test-cases-and-results}

\subsubsection{\texorpdfstring{\textbf{Test Case 1: User
Registration}}{Test Case 1: User Registration}}\label{test-case-1-user-registration}

{\def\LTcaptype{none} % do not increment counter
\begin{longtable}[]{@{}
  >{\raggedright\arraybackslash}p{(\linewidth - 2\tabcolsep) * \real{0.2222}}
  >{\raggedright\arraybackslash}p{(\linewidth - 2\tabcolsep) * \real{0.6111}}@{}}
\toprule\noalign{}
\begin{minipage}[b]{\linewidth}\raggedright
\textbf{Parameter}
\end{minipage} & \begin{minipage}[b]{\linewidth}\raggedright
\textbf{Details}
\end{minipage} \\
\midrule\noalign{}
\endhead
\bottomrule\noalign{}
\endlastfoot
Input & Username + Password \\
Expected Output & Account successfully created \\
Result & Passed --- user was created and stored in database \\
\end{longtable}
}

\textbf{Observation:}\\
Passwords were stored \textbf{only as bcrypt hashes}, not raw text
ensures security.

\subsubsection{\texorpdfstring{\textbf{Test Case 2: User Login with
Valid
Credentials}}{Test Case 2: User Login with Valid Credentials}}\label{test-case-2-user-login-with-valid-credentials}

{\def\LTcaptype{none} % do not increment counter
\begin{longtable}[]{@{}
  >{\raggedright\arraybackslash}p{(\linewidth - 2\tabcolsep) * \real{0.2222}}
  >{\raggedright\arraybackslash}p{(\linewidth - 2\tabcolsep) * \real{0.4722}}@{}}
\toprule\noalign{}
\begin{minipage}[b]{\linewidth}\raggedright
\textbf{Parameter}
\end{minipage} & \begin{minipage}[b]{\linewidth}\raggedright
\textbf{Details}
\end{minipage} \\
\midrule\noalign{}
\endhead
\bottomrule\noalign{}
\endlastfoot
Input & Registered Username + Correct Password \\
Expected Output & Login success and JWT token returned \\
Result & Passed \\
\end{longtable}
}

\textbf{Observation:}\\
JWT token successfully stored in browser session used for
authentication.

\subsubsection{\texorpdfstring{\textbf{Test Case 3: User Login with
Invalid Credentials (IDS
Trigger)}}{Test Case 3: User Login with Invalid Credentials (IDS Trigger)}}\label{test-case-3-user-login-with-invalid-credentials-ids-trigger}

\textbar{} Action \textbar{} User enters wrong password repeatedly
\textbar{}\\
\textbar{} Expected Behavior \textbar{} System blocks IP after threshold
attempts \textbar{}\\
\textbar{} Result \textbar{} Passed \textbar{}

\textbf{IDS Response Example Output:}

403 Forbidden -- Too many failed login attempts

Your IP has been temporarily blocked due to suspicious activity.

This confirms the \textbf{Intrusion Detection System is working
correctly.}

\subsubsection{\texorpdfstring{\textbf{Test Case 4: AES Key
Exchange}}{Test Case 4: AES Key Exchange}}\label{test-case-4-aes-key-exchange}

{\def\LTcaptype{none} % do not increment counter
\begin{longtable}[]{@{}
  >{\raggedright\arraybackslash}p{(\linewidth - 2\tabcolsep) * \real{0.2222}}
  >{\raggedright\arraybackslash}p{(\linewidth - 2\tabcolsep) * \real{0.5556}}@{}}
\toprule\noalign{}
\begin{minipage}[b]{\linewidth}\raggedright
\textbf{Parameter}
\end{minipage} & \begin{minipage}[b]{\linewidth}\raggedright
\textbf{Details}
\end{minipage} \\
\midrule\noalign{}
\endhead
\bottomrule\noalign{}
\endlastfoot
Action & Client sends encrypted AES session key to server \\
Expected Output & Server successfully decrypts and stores AES key \\
Result & Passed \\
\end{longtable}
}

\textbf{Observation:}\\
Man-in-the-middle inspection showed \textbf{AES key never transmitted in
plaintext} secure.

\subsubsection{\texorpdfstring{\textbf{Test Case 5: Sending a
Message}}{Test Case 5: Sending a Message}}\label{test-case-5-sending-a-message}

\textbar{} Action \textbar{} User A sends ``Hello'' to User B
\textbar{}\\
\textbar{} Expected Output \textbar{} Message delivered instantly and
encrypted \textbar{}\\
\textbar{} Result \textbar{} Passed \textbar{}

\textbf{Observed in Browser Network Tab:}

Ciphertext example: 1IqR4XFWLcuPMw==

IV: mlwRP1Lgj8SXIao9

No plaintext \textbf{ever appears in network logs}.

\subsubsection{\texorpdfstring{\textbf{Test Case 6: Real-Time Delivery
via
WebSocket}}{Test Case 6: Real-Time Delivery via WebSocket}}\label{test-case-6-real-time-delivery-via-websocket}

\textbar{} Action \textbar{} Message typing and sending while both users
online \textbar{}\\
\textbar{} Expected Output \textbar{} Message visible instantly on
recipient screen \textbar{}\\
\textbar{} Result \textbar{} Passed \textbar{}

\textbf{Latency Observed:}\\
\textless{} 100 ms (near real-time)

\subsection{\texorpdfstring{\textbf{9.2 Output
Screenshots}}{9.2 Output Screenshots}}\label{output-screenshots}

{\def\LTcaptype{none} % do not increment counter
\begin{longtable}[]{@{}
  >{\raggedright\arraybackslash}p{(\linewidth - 4\tabcolsep) * \real{0.2083}}
  >{\raggedright\arraybackslash}p{(\linewidth - 4\tabcolsep) * \real{0.4306}}
  >{\raggedright\arraybackslash}p{(\linewidth - 4\tabcolsep) * \real{0.3194}}@{}}
\toprule\noalign{}
\begin{minipage}[b]{\linewidth}\raggedright
\textbf{Screenshot No.}
\end{minipage} & \begin{minipage}[b]{\linewidth}\raggedright
\textbf{Description}
\end{minipage} & \begin{minipage}[b]{\linewidth}\raggedright
\textbf{Where to Capture}
\end{minipage} \\
\midrule\noalign{}
\endhead
\bottomrule\noalign{}
\endlastfoot
Fig 9.1 & User Registration Page & Login.jsx screen \\
Fig 9.2 & Successful Login + Session Key Setup & On login success
response \\
Fig 9.3 & Chat Window (Alice sending message to Bob) & Chat.jsx
screen \\
Fig 9.4 & Real-time message reception & Bob's window receiving
message \\
Fig 9.5 & IDS Block Warning & Attempt wrong passwords repeatedly \\
\end{longtable}
}

\includegraphics[width=6.17569in,height=4.05556in]{/Users/sj/Desktop/devsprint/report-engine/workspace/32a2bd09-5eed-4bc4-8ba4-8d9fc92e7564/media/image7.png}

\textbf{Figure 9.1}: Registration form interface for new user sign-up.

\includegraphics[width=5.99931in,height=3.77847in]{/Users/sj/Desktop/devsprint/report-engine/workspace/32a2bd09-5eed-4bc4-8ba4-8d9fc92e7564/media/image8.png}\textbf{Figure
9.2}: Successful login and secure session establishment.

\includegraphics[width=5.99653in,height=3.44097in]{/Users/sj/Desktop/devsprint/report-engine/workspace/32a2bd09-5eed-4bc4-8ba4-8d9fc92e7564/media/image9.png}\textbf{Figure
9.3}: Encrypted chat interface with Tailwind Dark UI.

\includegraphics[width=5.99653in,height=3.44444in]{/Users/sj/Desktop/devsprint/report-engine/workspace/32a2bd09-5eed-4bc4-8ba4-8d9fc92e7564/media/image10.png}

\textbf{Figure 9.4}: Real-time message update using WebSocket.

\includegraphics[width=5.99653in,height=3.46111in]{/Users/sj/Desktop/devsprint/report-engine/workspace/32a2bd09-5eed-4bc4-8ba4-8d9fc92e7564/media/image11.png}

\textbf{Figure 9.5}: Intrusion Detection System blocking suspicious
login attempts.

\subsection{\texorpdfstring{\textbf{9.3 Result
Analysis}}{9.3 Result Analysis}}\label{result-analysis}

The system successfully demonstrated:

{\def\LTcaptype{none} % do not increment counter
\begin{longtable}[]{@{}
  >{\raggedright\arraybackslash}p{(\linewidth - 4\tabcolsep) * \real{0.2500}}
  >{\raggedright\arraybackslash}p{(\linewidth - 4\tabcolsep) * \real{0.3333}}
  >{\raggedright\arraybackslash}p{(\linewidth - 4\tabcolsep) * \real{0.3889}}@{}}
\toprule\noalign{}
\begin{minipage}[b]{\linewidth}\raggedright
\textbf{Security Objective}
\end{minipage} & \begin{minipage}[b]{\linewidth}\raggedright
\textbf{How Achieved}
\end{minipage} & \begin{minipage}[b]{\linewidth}\raggedright
\textbf{Result}
\end{minipage} \\
\midrule\noalign{}
\endhead
\bottomrule\noalign{}
\endlastfoot
Confidentiality & AES-256 Encryption & Messages remained unreadable to
outsiders \\
Authentication & JWT + bcrypt & Only authorized users accessed system \\
Integrity & AES-GCM Auth Tag & Any message tampering detected \\
Availability & WebSocket real-time delivery & Chat remained responsive
and stable \\
Intrusion Resistance & IDS Login Monitoring & Brute-force attempts
blocked \\
\end{longtable}
}

The system meets the goals of a secure messaging platform. Performance
remained smooth under normal usage, and security features operated
effectively to prevent unauthorized access.

\subsection{\texorpdfstring{\textbf{9.4 Conclusion for Testing
Phase}}{9.4 Conclusion for Testing Phase}}\label{conclusion-for-testing-phase}

The testing results indicate that the \textbf{Secure Messaging System is
functionally correct, secure, and reliable}.\\
All major features performed as expected with no data leakage, UI
failure, or performance bottlenecks observed during testing. The
cryptographic components successfully protected message confidentiality
and integrity, while the IDS safeguarded system access against attack
attempts.

\section{\texorpdfstring{\textbf{10. Conclusion and Future
Enhancements}}{10. Conclusion and Future Enhancements}}\label{conclusion-and-future-enhancements}

\subsection{\texorpdfstring{\textbf{10.1
Conclusion}}{10.1 Conclusion}}\label{conclusion}

The Secure Messaging System developed in this project successfully
demonstrates how modern cryptographic techniques and real-time
communication protocols can be combined to provide \textbf{confidential,
authenticated, and reliable} digital communication. The integration of
\textbf{RSA-based key exchange}, \textbf{AES-GCM symmetric encryption},
and \textbf{WebSocket-based real-time message delivery} ensures that
messages remain private and cannot be intercepted or modified during
transmission.

Additionally, the implementation of a \textbf{lightweight Intrusion
Detection System (IDS)} enhances system security by detecting and
mitigating brute-force attacks, ensuring that only authorized users can
access their accounts. The usage of \textbf{bcrypt hashing for password
security} and \textbf{JWT for session authentication} further reinforces
the trust and safety of the communication environment.

Overall, the system successfully meets the core objectives of
\textbf{Confidentiality, Integrity, Authentication, and Availability
(CIA)}, and effectively illustrates the practical use of cryptographic
mechanisms in secure communication applications. The project also
demonstrates a clear alignment with real-world messaging systems such as
WhatsApp, Signal, Telegram, and secure enterprise messaging platforms.

\subsection{\texorpdfstring{\textbf{10.2 Future
Enhancements}}{10.2 Future Enhancements}}\label{future-enhancements}

While the system performs efficiently for academic and small-scale
practical use, several enhancements can be incorporated to extend
scalability, security, and usability:

{\def\LTcaptype{none} % do not increment counter
\begin{longtable}[]{@{}
  >{\raggedright\arraybackslash}p{(\linewidth - 4\tabcolsep) * \real{0.2192}}
  >{\raggedright\arraybackslash}p{(\linewidth - 4\tabcolsep) * \real{0.4932}}
  >{\raggedright\arraybackslash}p{(\linewidth - 4\tabcolsep) * \real{0.2603}}@{}}
\toprule\noalign{}
\begin{minipage}[b]{\linewidth}\raggedright
\textbf{Area of Enhancement}
\end{minipage} & \begin{minipage}[b]{\linewidth}\raggedright
\textbf{Description}
\end{minipage} & \begin{minipage}[b]{\linewidth}\raggedright
\textbf{Benefit}
\end{minipage} \\
\midrule\noalign{}
\endhead
\bottomrule\noalign{}
\endlastfoot
\textbf{True End-to-End Encryption (E2EE)} & Currently, the server
re-encrypts messages. In future, users can exchange encryption keys
directly using Diffie-Hellman or Public Key Fingerprinting. & Server
will not be able to access plaintext at any stage. \\
\textbf{Database Encryption} & Encrypting stored message ciphertexts
along with IVs using a master key. & Protects message data if server
storage is compromised. \\
\textbf{Key Rotation / Ephemeral Keys} & Regular generation and refresh
of AES session keys. & Improves forward secrecy and resilience against
key leaks. \\
\textbf{Group Chat Support} & Implement message fan-out encryption for
multiple recipients. & Enables use in teams or organizational
environments. \\
\textbf{File / Image Secure Transfer} & Extend AES encryption to binary
media streams. & Makes system more practical for real-world use. \\
\textbf{Advanced IDS / AI Threat Detection} & Machine learning models to
analyze abnormal login or network behavior. & Provides proactive cyber
threat intelligence. \\
\textbf{Mobile Application Interface} & Develop Android/iOS version
using React Native. & Increases accessibility and usability. \\
\end{longtable}
}

\section{\texorpdfstring{\textbf{10.3 Final
Remarks}}{10.3 Final Remarks}}\label{final-remarks}

This project demonstrates how secure communication can be implemented
effectively using open-source tools and well-established cryptographic
principles. By combining encryption, real-time networking, and threat
monitoring, the project not only fulfills curriculum objectives but also
provides a foundation that can be enhanced into a professionally
deployable secure messaging platform.

It highlights the importance of:

\begin{itemize}
\item
  Designing security early in software systems
\item
  Protecting data privacy as a fundamental right
\item
  Continual monitoring to detect evolving cyber threats
\end{itemize}

With further enhancements, the system can evolve into a
\textbf{production-grade secure communication application} suitable for
real-world usage.

\section{}\label{section-30}

\section{}\label{section-31}

\section{\texorpdfstring{\textbf{11. Future
Enhancements}}{11. Future Enhancements}}\label{future-enhancements-1}

Although the Secure Messaging System fulfills core security and
communication requirements, several enhancements can further improve
efficiency, scalability, and robustness. The following improvements may
be considered for future development:

\begin{itemize}
\item
  \textbf{True End-to-End Encryption (E2EE)}\\
  Currently, the server re-encrypts messages before forwarding them. In
  future versions, the sender and recipient can exchange encryption keys
  directly using \textbf{Diffie--Hellman} or \textbf{Signal Protocol},
  ensuring that even the server cannot access plaintext messages at any
  point.
\item
  \textbf{Group Chat and Broadcast Support}\\
  Extending the encryption logic to support secure \textbf{multi-user
  chats} can allow communication in teams and organizations. Each group
  would maintain its own secure session key.
\item
  \textbf{Encrypted File and Media Sharing}\\
  The system can be enhanced to support secure transfer of
  \textbf{images, audio, video, and documents} using AES encryption on
  binary data streams.
\item
  \textbf{Cloud Deployment and Scalability}\\
  Deploying the system on cloud platforms such as AWS, Azure, or GCP
  with load balancing can support thousands of concurrent users.
\item
  \textbf{Cross-Platform Mobile Applications}\\
  Implementing mobile versions using \textbf{React Native} or
  \textbf{Flutter} would make the system accessible across Android and
  iOS devices.
\item
  \textbf{AI-Based Intrusion Detection}\\
  The IDS can be enhanced to detect complex attack patterns using
  \textbf{machine learning} models trained on login and network behavior
  analytics.
\item
  \textbf{Database-Level Encryption}\\
  Encrypting stored ciphertext and metadata in the database using a
  server-wide master key would further strengthen data-at-rest security.
\end{itemize}

\section{\texorpdfstring{\textbf{12.
References}}{12. References}}\label{references}

\begin{enumerate}
\def\labelenumi{\arabic{enumi}.}
\item
  William Stallings, \emph{Cryptography and Network Security: Principles
  and Practices}, Pearson Education.
\item
  Atul Kahate, \emph{Cryptography and Network Security}, McGraw Hill
  Education.
\item
  Kevin Mandia, Chris Prosise, Matt Pepe, \emph{Incident Response and
  Computer Forensics}, McGraw Hill.
\item
  NIST, ``Advanced Encryption Standard (AES) Specification,'' Federal
  Information Processing Standard (FIPS) Publication 197.
\item
  Rivest, R., Shamir, A., Adleman, L., ``A Method for Obtaining Digital
  Signatures and Public-Key Cryptosystems,'' MIT, 1978.
\item
  FastAPI Documentation --- \url{https://fastapi.tiangolo.com}
\item
  Web Crypto API Documentation ---
  \url{https://developer.mozilla.org/en-US/docs/Web/API/Web_Crypto_API}
\item
  React Official Documentation --- \url{https://react.dev}
\item
  SQLite Database Guide --- \url{https://www.sqlite.org}
\item
  OWASP Foundation, ``Authentication and Password Management Best
  Practices,'' \url{https://owasp.org}
\end{enumerate}

\subsection{}\label{section-32}


\end{document}